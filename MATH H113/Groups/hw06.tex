\subsection{Homework 6}
1. First, we prove that every subgroup of $\Z$ is isomorphic to either $\Z$ or the trivial subgroup $\<0\>$. Suppose a subgroup $K$ of $\Z$ is not $\<0\>$. Then there exists a nonzero element in $\Z$. Consider the smallest positive element $k_0 \neq 0$ in $\Z$, we claim that $K = \<k_0\>$. Given any other element $k \in K$, we have by the division algorithm that $k = qk_0 + r$, for $0 \leq r < k_0$. But since $k_0$ was assumed to be the smallest positive element of $K$, $r$ must be 0. Then $k = qk_0$ and so $K$ is indeed generated by $k$. It follows that $K \approx \Z$ since we can construct the isomorphism given by mapping $k$ onto 1.

Now, we move on to $\Z^2$. Let $K$ be a subgroup of $\Z^2$, and $X$ be the set of first components of elements of $K$. In particular, we have $x \in X$ if $(x, y) \in K$. Since $K$ is a group in itself, $-(x, y) = (-x, -y) \in K$, so $-x \in X$. Furthermore, if we have $x, x' \in X$, then $(x, y) + (x', y') = (x + x', y + y') \in K$, so $x + x' \in X$. Thus $X$ is a subgroup of $\Z$. So either $X = \<0\>$ or $X = \<x_0\>$ for some $x_0$. In the former case, we obtain an isomorphism through projecting $K$ onto its second component and see that it is isomorphic to some subgroup of $\Z$, and hence isomorphic to either $\<0\>$ or $\Z$, in which case we are done. 

So assume the latter case, where $X = \<x_0\>$. In order for $x_0 \in X$, we must have that $(x_0, y_0) \in K$ for some integer $y_0$. Let $(x, y) \in K$. Since $x_0$ generates all elements of $X$, we have that
\[
    (x, y) = m(x_0, y_0) + (0, y - my_0).
\]
We claim that the set of $(0, y - my_0)$ for all possible $y$ and $m$ forms a subgroup $K'$ of $\Z^2$ that is isomorphic to $\Z$, since we can obtain an isomorphism by projecting all such elements onto the second component. If $(0, y - my_0) \in K'$, then we can take $-(x, y) = -m(x_0, y_0) - (0, y - my_0)$, and so $-(0, y - my_0) \in K'$. Furthermore, if $(x', y') \in K$ so that $(x', y') = m'(x_0, y_0) + (0, y' - m'y_0)$ and $(0, y' - m'y_0) \in K'$, then we have $(x, y) + (x', y') = (m + m')(x_0, y_0) + (0, (y + y') - (m + m')y_0)$ and so $(0, (y + y') - (m + m')y_0) \in K'$. It follows that $K'$ is a subgroup isomorphic to subgroup of $\Z$, so it is generated by at most one element. Then since $X$ is generated by one element, we have that $K$ is generated by at most two elements. If it is generated by two elements $v_1$ and $v_2$ then we can construct the isomorphism mapping $v_1 \mapsto (1, 0)$ and $v_2 \mapsto (0, 1)$, from which we get that $K \approx \Z^2$, which is trivial. In the other case, we get that $K \approx \<v\> \approx \Z$ for some nonzero element $v \in \Z$.

It follows that any nontrivial subgroup of $\Z^2$ must be isomorphic to $\Z$. 


2. By the theorem of classification of finite abelian groups, we have that all abelian groups of order 16 must be isomorphic to one of the following:
\[
    \Z_{16} \qquad \Z_8\oplus\Z_2 \qquad \Z_4\oplus\Z_4 \qquad \Z_4\oplus\Z_2\oplus\Z_2 \qquad \Z_2\oplus\Z_2\oplus\Z_2\oplus\Z_2
\]

Now, for $\Z_{17}^\times$, note that it is generated by 3, and therefore cyclic. In particular, if we write out the sequence $(3^k)$ mod 17, we get
\[
3, 9, 10, 13, 5, 11, 16, 14, 8, 7, 4, 12, 2, 6,
\]
and so $\Z_{17}^\times = \<3\> \approx \Z_{16}$.

Next, for $\Z_{32}^\times$, consider the mapping $\phi:\Z_8\oplus\Z_2 \to \Z_{32}^\times$ given by $(a, b) \mapsto 3^a(-1)^b$ (mod 32). Since $\phi(a, b) + \phi(a', b') = 3^a(-1)^b3^{a'}(-1)^{b'} = 3^{a + a'}(-1)^{b + b'} = \phi((a, b) + (a', b'))$, we see that $\phi$ is a homomorphism. Now, we list out the elements to show there is a bijection between group elements:
\[
\begin{tabular}{c|c|c|c|c|c|c|c|c}
& $(1, b)$ & $(2, b)$ & $(3, b)$ & $(4, b)$ & $(5, b)$ & $(6, b)$ & $(7, b)$ & $(0, b)$ \\
\hline
$(a, 0)$ & 3 & 9 & 27 & 17 & 19 & 25 & 11 & 1 \\
\hline
$(a, 1)$ & 29 & 23 & 5 & 15 & 13 & 7 & 21 & 31
\end{tabular}
\]
Since 3 has order 8 and -1 has order 2, it follows that $\Z_{32}^\times \approx \Z_8\oplus\Z_2$.

3. Recall the proof of classification of finite abelian groups using young tableaux. Let $H$ be a subgroup of $A$ that is isomorphic to $B$. Suppose $h_i$ is the generator element of the cyclic group $\Z_{p^{b_i}}$ in the direct sum of $H$. Then $H$ (and $B$) have the young diagram:
\[
\ytableausetup{boxsize=3.8em}
\begin{ytableau}
p^{b_1 - 1}h_1 & \dots & & \dots & h_1 \\
p^{b_2 - 1}h_2 & \dots & h_2 \\
\vdots & \vdots \\
h_n
\end{ytableau}
\]
Let $r_1, r_2, r_3, \dots$ be the heights of the first, second, third, etc. columns of the above diagram. Now recall the set $H(k) = \{h \in H | p^kh = 0\}$. Then we have that $|H(k)| = p^{r_1 + \dots + r_k}$ (this we have shown in our proof for classification of finite abelian groups). In particular, each square contributes $p$ possibilities (for coefficients in linear combinations), and $H(1)$ corresponds to those in the first column, $H(2)$ to those in the second, and so on.

Now, consider the sets $A(k) = \{a \in A | p^ka = 0\}$, defined equivalently to $H(k)$. Since $H$ is a subgroup of $A$, we know that if $h \in H(k)$, then certainly $p^kh = 0$ and so $h \in A(k)$. So $H(k) \subseteq A(k)$ for each $k$, and so $|A(k)| \geq |H(k)| = p^{r_1 + \dots + r_k}$. This implies that in the young diagram of $A$, column $i$ must have at least $r_i$ boxes. But then if we observe the diagram row-wise, we see that each row of $A$ must be at least as long as the corresponding row of $H$. Thus $A$ can be represented as a direct sum of cyclic groups where the orders $a_i$ are each at least as big as the corresponding order $b_i$ of the orders of $B$'s cyclic group decomposition.

4. First, consider the case where $G$ is abelian. By the classification theorem, we have the following three possibilities:
\[
\Z_8 \qquad \Z_4\oplus\Z_2 \qquad \Z_2\oplus\Z_2\oplus\Z_2
\]

Now, consider the case where $G$ is non-abelian. Clearly the quaternions is such a group. In particular, we have the symbols $\pm1, \pm i, \pm j, \pm k$, satisfying the relations $i^2 = j^2 = k^2 = ijk = -1$ and $ij = k$, $jk = i$, $ki = j$, $ji = -k$, $kj = -i$, and $ik = -j$. The quaternions are not abelian since $ij = k \neq -k = ji$.

The other non-abelian group of order 8 is the dihedral group $D_4$ consisting of rotations and reflections that preserve the square. It is not abelian since a reflection followed by a $90^\circ$ clockwise rotation is not the same as first a $90^\circ$ clockwise rotation followed by the same reflection.
