\section{Vector Spaces}

%%%%%%%%%%%%%%%%%%%%%%%%%%%%%%%%%%%%%%%%%%%%%%%%%%%%%%%%%
%%%%%%%%%%%%%%%%% 1.1 PRELIMINARIES %%%%%%%%%%%%%%%%%%%%%
%%%%%%%%%%%%%%%%%%%%%%%%%%%%%%%%%%%%%%%%%%%%%%%%%%%%%%%%%
\subsection{Preliminaries}

%%%%%%%%%%%%%%%%%%%% 1.1.1 %%%%%%%%%%%%%%%%%%%%%%%%%%%%
\subsubsection{Sets}
\textbf{Definition} (Power Set). The \textbf{\textit{power set}} of a set $S$ is the collection, or \textbf{\textit{family}}, of subsets of $S$, denoted $P(S)$. E.g. the family of open intervals in $\mathbb{R}$ is a subset of $P(\mathbb{R})$.

\textbf{Definition} (Cartesian Product). For two sets $X$ and $Y$, $X\times Y:=\{(x,y)\mid x\in X, y\in Y\}$, also known as the \textbf{\textit{Cartesian product}} of the factors $X$ and $Y$. Denote $X^n$ to be the Cartesian product of a set with itself $n$ times.

\textbf{Definition} (Relation). A \textbf{\textit{relation}} between a set $X$ and $Y$ is any subset of $X\times Y$. A relation between a set and itself is said to be a relation \textit{on} $X$. 

\textbf{Definition} (Partial Order). A \textbf{\textit{partial order}} on set $S$ is a relation defined by $P\subseteq S\times S$ such that the following properties hold:
\begin{enumerate}
    \item \textbf{(antisymmetry)} $(a,b)\in P\wedge(b,a)\in P\implies a=b$.
    \item \textbf{(transitivity)} $(a,b)\in P\wedge(b,c)\in P\implies(a,c)\in P$.
    \item \textbf{(reflexivity)} $\forall a\in S$, $(a,a)\in P$.
\end{enumerate}
Typically, we write $a\leq b$ if $(a,b)\in P$. Furthermore we write $a<b$ if $a\leq b$ and $a\neq b$. So then the above properties become:
\begin{enumerate}
    \item \textbf{(antisymmetry)} $a\leq b\wedge b\leq a\implies a=b$.
    \item \textbf{(transitivity)} $a\leq b\wedge b\leq c\implies a\leq c$.
    \item \textbf{(reflexivity)} $a\leq a$.
\end{enumerate}
\textbf{Definition} (Poset). If there exists a partial order on $S$, then we say $S$ is a \textbf{\textit{partially-ordered-set}}, or \textbf{\textit{poset}}. If $a$ and $b$ are elements of a poset and $a\leq b$ or $b\leq a$, then we say that $a$ and $b$ are \textbf{\textit{comparable}}. Otherwise they are \textbf{\textit{incomparable}}.

\textbf{Definition} (Max \& Min). Suppose $P$ is a partially ordered set and $X\subseteq P$. Then $X$ inherits a partial order from $P$ (think subgraphs). For an element $u\in P$ s.t. $u\geq x$ for all $x\in X$, $u$ is an \textbf{\textit{upper bound}} of $X$. For an element $M\in X$ s.t. $M\geq x$ for all $x\in X$, $M$ is a \textbf{\textit{maximal}} element. Similarly, for an element $l\in P$ s.t. $l\leq x$ for all $x\in X$, $l$ is a \textbf{\textit{lower bound}} of $X$. For an element $m\in X$ s.t. $m\leq x$ for all $x\in X$, $m$ is a \textbf{\textit{minimal}} element.

\textbf{Definition} ("Toset"). A partial order in which every pair of elements is comparable is called a \textbf{\textit{total order}}. A set with a total order is known as a \textbf{\textit{totally ordered set}}. A \textbf{\textit{chain}} is a totally ordered subset of a partially ordered set

%%%%%%%%%%%%%%%%%% 1.1.2 %%%%%%%%%%%%%%%%%%%%%%%%%%
\subsubsection{Functions}
\textbf{Definition} (Function). A \textbf{\textit{function}} $f$ is defined to be a relation between sets $X$ and $Y$ s.t. for each $x\in X$ there exists exactly one $y\in Y$ such that the pair $(x,y)$ is included in the relation defined by $f$. The set $X$ is the \textbf{\textit{domain}} of $f$ and $Y$ is the \textbf{\textit{codomain}} of $f$. The symbol $\mapsto$ reads "maps to". Also, the collection of \textit{all} functions $f:X\to Y$ is denoted $\mathcal{F}(X,Y)$.

\textbf{Definition} (Composition). If $f:X\to Y$ and $g:Y\to Z$, then their \textbf{\textit{composition}} $g\circ f:X\to Z$ is defined by $(g\circ f)(x)=g(f(x))$. Know how to draw "commutative diagrams".

\textbf{Definition} (Power Functions). Given a function $f:X\to Y$, $P(f):P(X)\to P(Y)$ is equivalent to the function that maps a subset of $X$ to a subset of $Y$. Thus $P$ is in itself a function, $P: \mathcal{F}(X,Y)\to \mathcal{F}(P(X),P(Y))$. Also, $P^{-1}(f):P(Y)\to P(X)$ is the power function of $f$ that sends a subset in $Y$ to its preimage.

\textbf{Definition} (Image, Range, Preimage). Given a set $S\subseteq X$ and a function $f:X\to Y$, the \textbf{\textit{image}} is the set $f(S):=\{f(x)\mid x\in S\}$. If $S=X$, then $f(S)$ is the \textbf{\textit{range}} of $f$. The range is a subset of the codomain. For a set $T\subseteq Y$, the \textbf{\textit{preimage}} is the set $f^{-1}(T):=\{x\in X\mid f(x)\in T\}$. 

\textbf{Definition} (Function Inverse). Given a function $f:X\to Y$, $f$ is invertible if for each $y\in Y$, there exists a unique $x\in X$ s.t. $f^{-1}(y)=x$. In particular, it is invertible if its range is equivalent to its codomain and the preimage of each singleton in its range is a singleton in its domain.

\textbf{Definition} (Injective, Surjective, Bijective). A function $f:X\to Y$ always satisfies that if $x=y$, then $f(x)=f(y)$. $f$ is \textbf{\textit{injective}} is $f(x)=f(x')$ implies $x=x'$. Such a function is called a \textit{one-to-one} function. $f$ is \textbf{\textit{surjective}} if for all $y\in Y$, there exist an $x\in X$ s.t. $f(x)=y$. This property is also called \textit{onto}. If a function is both injective and surjective, then it is called \textbf{\textit{bijective}}, otherwise known as a \textbf{\textit{one-to-one correspondence}}.

\textbf{Proposition}. A function $f:X\to Y$ is invertible if and only if it is bijective.

\textbf{Definition} (Inclusion, Restriction, Extension). For a set $A\subseteq B$, the \textbf{\textit{inclusion}} function is defined by $\iota_{A,B}=\iota:A\to B$ s.t. for an $a\in A$, $\iota(a)=a$. 

On the other hand, for a set $S\subseteq A$, the \textbf{\textit{restriction}} of $f$ to $S$ for a function $f:A\to B$ is $f\mid_S:S\subseteq A\to B$ s.t. $f\mid_S(x)=f(x)$ for $x\in S$. 

For a set $T$ s.t. $B\subseteq T$, we can \textbf{\textit{extend}} the codomain of $f$ to $T$ by composition with $\iota:B\to T$. That is, $\iota\circ f:A\to T$ satisfies $\iota(f(x))=f(x)$ for all $x\in A$.

%%%%%%%%%%%%%%%%%%%%% 1.1.3 %%%%%%%%%%%%%%%%%%%%%%%%%%%%%
\subsubsection{Lists \& Sequences}
\textbf{Definition} (Lists \& Sequences). Let $X$ be a set and $s\in\mathbb{N}$. Then a \textbf{\textit{list}} $(x_1, x_2,..., x_n)$ is a function from $\{1,2,...,n\}\to X$ s.t. $i\mapsto x_i$. Order (permutation) matters and there can also be duplicity in lists. A \textbf{\textit{sequence}} is essentially an infinite list that takes $\mathbb{N}\to X$.

\textbf{Definition} (Countability). We say that a set is \textbf{\textit{countably infinite}} if there exists a one-to-one correspondence with $\mathbb{N}$. A set is \textbf{\textit{countable}} if it is either finite or countably infinite. An \textbf{\textit{uncountable}} set is one that is not countable. Two sets between which there exists a bijection are said to have the same cardinality.

\textbf{Definition} (Index Set). We say that a set $A$ indexes $S$ if $f:A\to S$ is a list or a sequence.

\textbf{Definition} (Dirac Delta Function). Let $S$ be a set. For $p\in S$, the \textbf{\textit{Dirac Delta function}} at $p$, also called the \textbf{\textit{indicator function}} at $p$, is the function $\delta_p:S\to \{0,1\}$ s.t. $\delta_p(s)=0$ if $s\neq p$ and $\delta_p(s)=1$ if $s=p$. The point $p$ is called the base point of $\delta_p$. Note also that there is a bijection between $S$ and $\delta_p$.

%%%%%%%%%%%%%%%%%%%%%%%% 1.1.4 %%%%%%%%%%%%%%%%%%%%%%%%%%
\subsubsection{Fields \& Complex Numbers}
An element of $\mathbb{F}(X\times X, X)$ is called a \textbf{\textit{binary operation}} on $X$. A binary operator $g:X\times X\to X$ is \textbf{\textit{commutative}} if $g(x_1, x_2)=g(x_2, x_1)$ for all $(x_1,x_2)\in X\times X$. A binary operation on $X$ is \textbf{\textit{associative}} if $g(x_1,g(x_2, x_3))=g(g(x_1,x_2),x_3)$ for all $x_1, x_2, x_3\in X$.

\textbf{Definition} (Group). A \textbf{\textit{group}} $G=(|G|,g)$ is the pairing of an underlying set $|G|$ with a binary operation $g:|G|\times|G|\to|G|$ satisfying the following three axioms:
\begin{enumerate}
    \item (associativity) $g$ is an associative operation on $|G|$.
    \item (identity) There exists an element $e\in|G|$ s.t. $g(e,a)=g(a,e)=a$ for all $a\in|G|$, known as the \textbf{\textit{identity}} element.
    \item (inverse) For every $a\in|G|$, there exists an element $a'\in|G|$ s.t. $g(a,a')=g(a',a)=e$.
\end{enumerate}
If $g$ is also a commutative operation on $|G|$, then $G$ is called an \textbf{\textit{abelian group}}.

\textbf{Definition} (Field). A \textbf{\textit{field}} $\mathbf{F}$ is a set $F$ equiped with two binary operations, "addition" and "multiplication", such that $(F,+)$ is an abelian group and $(F\setminus 0,\cdot)$ is an abelian group where "0" is the identity of $(F,+)$. Furthermore, for all $a, b, c\in F$, \textbf{\textit{distributivity of multiplication over addition}} holds:
\[
a\cdot(b+c)=a\cdot b + a\cdot c
\]
We write $\mathbf{F}$ as the triple $(F,+,\cdot)$, where $F$ is the underlying set of $\mathbf{F}$. In checking whether a set $F$ equipped with two binary operations is a field, we need to check whether addition and multiplication are "closed".

\textbf{Definition} (Characteristic). The \textbf{\textit{characteristic}} of a field is the smallest $n$ such that when "1" is "added" onto itself $n$ times, "0" is produced. If no such $n$ exists, then the characteristic is defined to be 0.

\textbf{Definition} (Polynomial). A \textbf{\textit{polynomial}} in one variable over a field $\mathbf{F}$ is defined to be a function $p:\mathbf{F}\to\mathbf{F}$ where
\[
p(z)=a_nz^n+a_{n-1}z^{n-1}+\ldots+a_1z+a_0
\]
for all $z\in\mathbf{F}$ and for some fixed $a_0,...,a_n\in\mathbf{F}$. Such a polynomial has similar constituent definitions to the classic definition of a polynomial.

\textbf{Definition} (Ordered Field). A field $\mathbf{F}$ together with a total order $\leq$ on $F$ is an \textbf{\textit{ordered field}} if the order satisfies the following properties:
\begin{enumerate}
    \item if $a\leq b$, then $a+c\leq b+c$;
    \item if $0\leq a$ and $0\leq b$, then $0\leq a\cdot b$.
\end{enumerate}
The sets $\mathbb{Q}$ and $\mathbb{R}$ are examples of ordered fields.


%%%%%%%%%%%%%%%%%%%%%%%%%%%%%%%%%%%%%%%%%%%%%%%%%
%%%%%%%% 1.2 DEFINITION OF VECTOR SPACE %%%%%%%%%
%%%%%%%%%%%%%%%%%%%%%%%%%%%%%%%%%%%%%%%%%%%%%%%%%
\subsection{Definition of a Vector Space}
A \textbf{\textit{vector space}} $V$ over a field $\mathbf{F}$ is a set $|V|$ together with the binary operation of \textbf{\textit{vector addition}}
\begin{align*}
    |V|\times|V|\to|V| \\
    (v,w)\mapsto v + w
\end{align*}
and \textbf{\textit{scalar multiplication}}
\begin{align*}
    \mathbf{F}\times|V|\to|V| \\
    (a,v)\mapsto av
\end{align*}
such that $(|V|,+)$ is an abelian group with the following axioms:
\begin{enumerate}
    \item Associativity of scalar multiplication
    \item Identity of scalar multiplication
    \item Distributivity over vector addition
    \item Distributivity over field addition
\end{enumerate}
Note that the elements of $\mathbf{F}$ are called \textit{scalars} whereas the elements of $|V|$ are called \textit{vectors}. So effectively, a vector space comes with 4 pieces of data: $(|V|, \mathbf{F}, \text{addition}, \text{scalar multiplication})$.



\newpage %%%%%% EXERCISES %%%%%%
\section{Homework 1}
 
Sundry? (that's some spicy vocab) :

I worked alone and TeX-ed my solutions as I solved the problems.

\begin{enumerate}
    %%%%%%%%%% TAUTOLOGY AND CONTRADICTIONS %%%%%%%%%
    \item\begin{enumerate}[(a)]
        % Part A
        \item \boxed{\text{Contradiction.}} \newline 
        \begin{tabular}{c|c|c|c|c}
            x & y & x$\Rightarrow$y & $\neg$y & (a) \\ \hline
            T & T & T & F & F \\ \hline
            T & F & F & T & F \\ \hline
            F & T & T & F & F \\ \hline
            F & F & T & T & F
        \end{tabular}
        % Part B
        \item \boxed{\text{Tautology.}} \newline
        \begin{tabular}{c|c|c|c}
            x & y & x$\vee$y & (b) \\ \hline
            T & T & T & T \\ \hline
            T & F & T & T \\ \hline
            F & T & T & T \\ \hline
            F & F & F & T
        \end{tabular}
        % Part C
        \item \boxed{\text{Tautology.}} \newline
        \begin{tabular}{c|c|c|c|c}
            x & y & x$\vee$y & x$\vee\neg$y & (c) \\ \hline
            T & T & T & T & T \\ \hline
            T & F & T & T & T \\ \hline
            F & T & T & F & T \\ \hline
            F & F & F & T & T
        \end{tabular}
        % Part D
        \item \boxed{\text{Tautology.}} \newline
        \begin{tabular}{c|c|c|c|c}
            x & y & x$\Rightarrow$y & x$\Rightarrow\neg$y & (d) \\ \hline
            T & T & T & F & T \\ \hline
            T & F & F & T & T \\ \hline
            F & T & T & T & T \\ \hline
            F & F & T & T & T
        \end{tabular}
        % Part E
        \item \boxed{\text{Neither.}} \newline
        \begin{tabular}{c|c|c|c|c}
            x & y & x$\vee$y & $\neg$(x$\wedge$y) & (d) \\ \hline
            T & T & T & F & F \\ \hline
            T & F & T & T & T \\ \hline
            F & T & T & T & T \\ \hline
            F & F & F & T & F
        \end{tabular}
        % Part F
        \item \boxed{\text{Contradiction.}} \newline
        \begin{tabular}{c|c|c|c|c|c}
            x & y & x$\Rightarrow$y & $\neg$x$\Rightarrow$y & $\neg$y & (f)\\ \hline
            T & T & T & T & F & F \\ \hline
            T & F & F & T & T & F \\ \hline
            F & T & T & T & F & F \\ \hline
            F & F & T & F & T & F
        \end{tabular}
    \end{enumerate}
    
    %%%%%%%%%%%%%% MISC LOGIC %%%%%%%%%%%%%%%%%%%%
    \item\begin{enumerate}[(a)]
        \item\begin{enumerate}[(i)]
            \item \boxed{\text{Possibly True.}} It can be false if there exists some other $y\neq4$ s.t. $G(3, y)$ is true. It is true simply when $G(3, 4)$ is true.
            \item \boxed{\text{Possibly True.}} It will be false when there exists an $x$ s.t. $G(x,3)$ is not true, and instead there exists a $y\neq3$ s.t. $G(x,y)$ is true. It is true simply when $G(x,3)$ is true for all $x$.
            \item \boxed{\text{Certainly True.}} Since $3\in\mathbb{R}$, there must exist a $y$ s.t. $G(3,y)$ is true.
            \item \boxed{\text{Certainly False.}} Since $3\in\mathbb{R}$, there exists at least one $y$ s.t. $\neg G(3,y)$ is false, so $\neg G(3,y)$ can't be true for all $y$.
            \item \boxed{\text{Possibly True.}} It is true if, let's say, $G(0, 4)$ were true (or as long as for some arbitrary $x$, $G(x, 4)$ is true). It is false when the only $G(x, 3)$ is true for all $x$ but $G(x,4)$ is false for all $x$.
        \end{enumerate}
        \item \boxed{(X \vee Y \vee Z)\wedge\neg((X\wedge Y)\vee(Y\wedge Z)\vee(X\wedge Z)).}
    \end{enumerate}
    
    %%%%%%%%% PROPOSITIONAL PRACTICE %%%%%%%%%
    \item\begin{enumerate}[(a)]
        \item \boxed{(\exists x\in\mathbb{R})(x\notin\mathbb{Q}).} Statement is \boxed{\text{True}}, as numbers like $\pi$ or $e$ are irrational real numbers.
        \item \boxed{(\forall x\in\mathbb{Z})((x\in \mathbb{N}\vee x<0)\wedge\neg(x\in\mathbb{N}\wedge x<0)).} Statement is \boxed{\text{False}}, as the integer 0 is not included in the set.
        \item \boxed{(n\in\mathbb{N})(6\mid n \Rightarrow ((2\mid n) \vee (3\mid n))).} Statement is \boxed{\text{True}} but weak. If 6 divides $n$ then both 2 \textit{and} 3 must divide $n$.
        \item All real numbers are also complex numbers. Statement is \boxed{\text{True}}, as all real numbers are essentially complex numbers of the form $a+bi$ where $b=0$.
        \item If an integer is divisible by 2 or divisible by 3, then it is divisible by 6. Statement is \boxed{\text{False}}. E.g. 2 is divisible by 2 \textit{or} 3 but neither are divisible by 6.
        \item If a natural number $x$ is greater than 7, then there exists natural numbers $a$ and $b$ that sum to $x$. Statement is \boxed{\text{True}}, as the smallest natural number is 1, so therefore all natural numbers at least 2 can be represented by the sum of two natural numbers.
    \end{enumerate}
    
    %%%%%%%%%%%% PROOFS %%%%%%%%%%%%%%%%%%
    \item\begin{enumerate}[(a)]
        \item \boxed{\text{Proof by Contraposition.}} The contrapositive states that if 10 divides $x$ or 10 divides $y$, then 10 divides $xy$. This is true because WLOG (without loss of generality) assume that 10 divides $x$, then $x=10a$ for some integer $a$. Then $xy=10ay$, and so 10 divides $xy$ and thus since the contrapositive is true, the original statement must also be true.
        \item I just did.
        \item The converse states that if 10 does not divide $x$ and 10 does not divide $y$ then 10 does not divide $xy$. However, if $x=2$ and $y=5$, then 10 divides $xy$. Thus the converse is \boxed{\text{False.}}
    \end{enumerate}
    
    %%%%%%%%%%% PROVE OR DISPROVE %%%%%%%%%%%%%%%%
    \item\begin{enumerate}[(a)]
        \item \boxed{\text{True.}} If $n$ is odd then $n^2$ is odd and $2n$ is even, and so the sum $n^2+2n$ must also be odd.
        \item \boxed{\text{True.}} If $x<y$, then the RHS would evaluate to $(x+y+(x-y))/2=x$. If $x>y$, then the RHS would evaluate to $(x+y-(x-y))/2=y$. therefore the statement is true.
        \item \boxed{\text{True.}} We proceed by examining two cases. Let $a$ and $b$ be any real numbers s.t. $a+b\leq10$. Case 1: if $a\leq7$, then we are done as one of the conditions on the right is satisfied. Case 2: if $a>7$, then let $a=7+k$ for some positive real number $k$. Then the original inequality simplifies to $k+b\leq3$, or $b<3$, which is satisfied by the second condition on the right, $b\leq3$. Thus the statement is always true.
        \item \boxed{\text{True.}} Assume for sake of contradiction that $r+1$ is rational. So $r+1=\frac{p}{q}$ for some integers $p$ and $q$. Then $r=\frac{p-q}{q}$. Since the RHS is a fraction whose numerator and denominator are both integers, it follows that $r$ is also a rational number $\Rightarrow\Leftarrow$. Thus by contradiction if $r$ is irrational, then so is $r+1$.
        \item \boxed{\text{False.}} Counterexample: let $n=6$. Then $10(6)^2=360\ngtr720=6!$.
    \end{enumerate}
    
    %%%%%%%%%% PRESERVING SET OPERATIONS %%%%%%%%%%
    \item\begin{enumerate}[(a)]
        \item On the LHS, we have $S_1=f^{-1}(A\cup B)=\{x\mid f(x)\in A\cup B\}$. On the RHS, we have $S_2=f^{-1}(A)\cup f^{-1}(B)=\{x\mid f(x)\in A\}\cup\{x\mid f(x)\in B\}$.

        If $a\in S_1$, then $f(a)\in A\cup B$. It then follows that $f(a)\in A \vee f(a)\in B$. Then $(a\in\{x\mid f(x)\in A\})\vee (a\in\{x\mid f(x)\in B\})$, which is equivalent to $a\in S_2$.
        
        On the other hand, if $a\in S_2$, then $f(a)\in A \vee f(a)\in B$. It follows that $f(a)\in A\cup B$, which means that $a\in S_1$.
        
        Thus $S_1=S_2$, and we are done.
        
        \item On the LHS, we have $S_1=f^{-1}(A\cap B)=\{x\mid f(x)\in A\cap B\}$. On the RHS, we have $S_2=f^{-1}(A)\cap f^{-1}(B)=\{x\mid f(x)\in A\}\cap\{x\mid f(x)\in B\}$.
        
        If $a\in S_1$, then $f(a)\in A\cap B$. It then follows that $f(a)\in A \wedge f(a)\in B$. Then $(a\in\{x\mid f(x)\in A\})\wedge (a\in\{x\mid f(x)\in B\})$, which is equivalent to $a\in S_2$.
        
        On the other hand, if $a\in S_2$, then $f(a)\in A \wedge f(a)\in B$. It follows that $f(a)\in A\cap B$, which means that $a\in S_1$.
        
        Thus $S_1=S_2$, and we are done.
        
        \item On the LHS, we have $S_1=f^{-1}(A\setminus B)=\{x\mid f(x)\in A\setminus B\}$. On the RHS, we have $S_2=f^{-1}(A)\setminus f^{-1}(B)=\{x\mid f(x)\in A\}\setminus\{x\mid f(x)\in B\}$.
        
        If $a\in S_1$, then $f(a)\in A\setminus B$. It then follows that $f(a)\in A \wedge f(a)\notin B$. Then $(a\in\{x\mid f(x)\in A\})\wedge(a\notin\{x\mid f(x)\in B\})$, which is equivalent to $a\in\{x\mid f(x)\in A\}\setminus\{x\mid f(x)\in B\}= S_2$.
        
        On the other hand, if $a\in S_2$, then $f(a)\in A \wedge f(a)\notin B$. It follows that $f(a)\in A\setminus B$, which means that $a\in S_1$.
        
        Thus $S_1=S_2$, and we are done.
        
        \item Let $S=f(A\cup B)$, so every element that is either in $A$ or in $B$ gets mapped to an element in $S$. Then the image space of both $f(A)$ and $f(B)$ lie in $S$. The same is true backwards. If an $y\in f(A)\cup f(B)$, then $x$ s.t $f(x)=y$ must have been mapped from the set $A$ or the set $B$, and so $x\in A\cup B$. Therefore $y\in S$. Thus the statement is indeed an equivalency.
        
        \item For $x\in A\cap B$, let $f(x)=y$, so $y\in f(A\cap B)$. Then, since $x\in A$ and $x\in B$, $y\in f(A)$ and $y\in f(B)$, so $y\in f(A)\cap f(B)$. Thus $f(A\cap B) \subseteq f(A)\cap f(B)$. Example where equality does not hold: Let $A=\{0\}$ and $B=\{1\}$. Also let $f(0)=f(1)=42$. Then $f(A\cap B)$ is the empty set whereas $f(A)\cap f(B)$ is the set $\{42\}$.
        
        \item For $y\in f(A)\setminus f(B)$, define $x$ s.t. $y=f(x)$. Then $x\in A$ and $x\notin B$, or $x\in A\setminus B$. Therefore $y\in f(A\setminus B)$. Example where equality does not hold: Let $A=\{0, 1\}$ and $B=\{1\}$. Also let $f(0)=f(1)=42$. Then $f(A\setminus B)=\{42\}\supset\{ \}=f(A)\setminus f(B)$. 
    \end{enumerate}
\end{enumerate}
\newpage
\textbf{1.14 Solution.} First, note that $\mathcal{C}([a,b],\mathbb{R})$ is nonempty. Since we can assume that the sum of two continuous functions is continuous, then we know that for two functions $c_1, c_2\in\mathcal{C}([a,b], \mathbb{R})$, the function $c_1+c_2$ must also be in $\mathcal{C}([a,b], \mathbb{R})$. Furthermore, since we can assume that a scalar multiple of a continuous function is also continuous, then we know that for any scalar $a$ and any function $c\in\mathcal{C}([a,b],\mathbb{R})$, the function $ac$ is also in $\mathcal{C}([a,b],\mathbb{R})$.

\textbf{1.15 Solution.} No, the set is not closed under addition and scalar multiplication. That is, for a polynomial $ax^2+bx+1$, $(ax^2+bx+1)+(ax^2+bx+1)= 2ax^2+2bx+2$ cannot be in the set because the constant term is 2, whereas the set only contains polynomials where the constant term is 1.

\textbf{1.16 Solution.} First, note that the set of all polynomials with degree $\leq n$ is nonempty. Next, note that any polynomial with degree $\leq n$ must also be in the set of all polynomials over $\mathbb{F}$. For two polynomials $p_1(x)$ and $p_2(x)$ of degree $\leq n$, the polynomial $p_1(x)+p_2(x)$ is also a polynomial of degree $\leq n$, since addition of corresponding coefficients is closed in the field $\mathbb{F}$. Finally, for a polynomial $p(x)$ of degree $\leq n$ and a scalar $a\in\mathbb{F}$, the polynomial $a(p(x))$ is also a polynomial of degree $\leq n$, since multiplication of the coefficients by a scalar is also closed in the field $\mathbb{F}$. Therefore it is indeed a subspace.

Yes, any set of polynomials under the condition that only a predetermined set of powers have terms with nonzero coefficients. For example, the set of polynomials $\{ax^8+bx^5+cx^2+d\mid a,b,c,d\in\mathbb{F}\}$ is a subspace of $\mathcal{P}(\mathbb{F})$.

\textbf{1.17 Solution.}
(a) Let $T$ and $U$ be two subspaces of $V$ over the field $\mathbb{F}$. Then since $T$ and $U$ are closed under scalar multiplication, (if we let the scalar be 0) the vector \textbf{0} in $V$ must also be in both $T$ and $U$, hence $\textbf{0}\in T\cap U$. This means that $T\cap U$ is nonempty. Now, suppose we have vectors $v_1,v_2\in T\cap U$, then $v_1$, $v_2$, and $v_1+v_2$ are all in $T$, and similarly $v_1$, $v_2$, $v_1+v_2$ are all in $U$ as well. Thus $v_1+v_2\in T\cap U$ and $T\cap U$ is closed under addition. Finally, suppose we have a vector $v\in T\cap U$ and a scalar $a\in\mathbb{F}$. Then it follows that since $v\in T$, $av\in T$, and similarly since $v\in U$, $av\in U$. Thus $av\in T\cap U$ and $T\cap U$ is closed under scalar multiplication. Therefore $T\cap U$ is indeed a subspace of $V$.

(b) The $x$-axis and $y$-axis are both subspaces of the $xy$-plane over the field of real numbers $\mathbb{R}$. However, their union is not a valid subspace of the $xy$-plane since it is not closed under addition.

(c) Assume for sake of contradiction that there is a subspace $S$ of $\mathbb{F}$ other than $\mathbb{F}$ and \textbf{0}. Then for vectors $a,b\in\mathbb{F}$, $a+b$ must be in $S$. Also, for a "scalar" $a\in\mathbb{F}$ and a "vector" $b\in\mathbb{F}$, the $ab$ must also be in $S$. But since all of these ``scalars'' and ``vectors'' are elements of the same set $\mathbb{F}$, the set $S$ is equivalent to the underlying set of $\mathbb{F}$, since $\mathbb{F}$ is closed under addition and multiplication operations. Thus the only subspaces of $\mathbb{F}$ are $\mathbb{F}$ itself and \textbf{0}, which is a subspace of all vector spaces.

\textbf{1.18 Solution.} First, $span(X)$ is always nonempty, since if $X$ were empty, then $span(X)$ would be the subspace $\{0\}$. Now, suppose we have vectors $v_1, v_2\in X$, then by the definition of $span(X)$, $v_1,v_2\in span(X)$ and therefore $v_1+v_2$ must also exist in $span(X)$. Thus $span(X)$ is closed under addition. Furthermore, suppose we have a vector $v\in X$ and a scalar $a\in\mathbb{F}$, then once again by the definition of $span(X)$, $v\in span(X)$ and so $av$ must also exist in $span(X)$. Thus $span(X)$ is also closed under scalar multiplication and is therefore a subspace of $V$.

\textbf{1.19 Solution.} We will show that $S(V)$ is an abelian group without inverses under the $\cap$ ("intersection") operation. 

Set intersection is associative. In particular, if we have an element $x$ in either $S_1\cap(S_2\cap S_3)$ or $(S_1\cap S_2)\cap S_3$, then it is true that $x$ is also in the other since both statements are the same as saying $x$ is in $S_1$, $S_2$, \textit{and} $S_3$.

Set intersection is also commutative. In particular, if an element $x$ is in either $S_1\cap S_2$ or $S_2\cap S_1$, then it is true that $x$ is also in the other since both statements are the same as saying $x$ is in $S_1$ \textit{and} $S_2$.

Finally, for every subspace $U\in S(V)$, the identity element under $\cap$ is $|V|$, since the intersection of original vector space with any subspace $U$ is just $U$ itself.

\textbf{1.20 Solution.} First, let $v_1+U=v_2+U$. Then if we add the additive inverse of $v_2$ to both sides, we get $v_1+U-v_2=v_2+U-v_2$. Since vector addition is commutative, we then have $v_1-v_2+U=U$. Then, since $U$ is a subspace, it must be closed under addition, and so  $v_1-v_2+\textbf{0}\in (v_1-v_2+U)=U$, therefore the vector $v_1-v_2$ must be in $U$.

Now, we prove the converse. Let $v_1-v_2\in U$. Then, since $U$ is a subspace, it must be closed under addition. Then the set $\{v_1-v_2+u\mid u\in U\}$ is equivalent to $\{u\mid u\in U\}$, so the affine subset $v_1-v_2+U$ is equivalent to the affine subset $U$. Finally, if we "add" $v_2$, which is the additive inverse of $-v_2$, we get that $v_2+U$ is an equivalent affine subset to $v_1+U$ by the commutativity of vector addition. 

Thus, $v_1+U=v_2+U$ if and only if $v_1-v_2\in U$.

\textbf{1.21 Solution.} First, let $v+U$ be a subspace of $V$. Then $v+U=\{v+u\mid u\in U\}$. Since we showed earlier that \textbf{0} is an element of every vector subspace, we let $u=\textbf{0}$ to get that $v\in v+U$. Then because $v+U$ is a subspace it must be closed under addition, and therefore $v+v\in\{v+u\mid u\in U\}$. But this can only happen if we let $u=v$, so therefore $v\in U$.

Now, we prove the converse. Let $v\in U$. Then since $U$ is closed under addition, we take the set $v+U=\{v+u\mid u\in U\}$ and deduce that it is the same set as $U$, which is a subspace. Therefore $v+U$ is a subspace of $V$, since $U$ is a subspace of $V$.

\textbf{1.22 Solution.} First, we know that $(\{v+U\mid v\in V\}, ``+")$ is an abelian group because the properties of commutativity and associativity carry over. Also, the identity element under ``+'' is just $U$, and the inverse of each $v+U$ is $-v+U$.

Associativity of scalar multiplication follows from $a(b(v+U))=abv+U=(ab)(v+U)$. The scalar identity is still the element $1\in\mathbb{F}$, such that $1(v+U)=v+U$.

Distributivity over vector addition also holds, since $a((v_1+U)``+"(v_2+U))=a(v_1+v_2+U)=a(v_1+v_2)+U=av_1+av_2+U=(av_1+U)``+"(av_2+U)=a(v_1+U)``+"a(v_2+U)$.

Finally, distributivity over field addition also holds, since $(a+b)(v+U)=(a+b)v+U=av+bv+U=(av+U)``+"(bv+U)=a(v+U)``+"b(v+U)$.

Thus $V\setminus U$ is a vector space since it satisfies all the vector space axioms.

\textbf{1.23 Solution.} Let $S$ be an affine subset of $V$. Then we can treat each of the $\lambda_i$'s as a "weight", such that the sum of the product of the vectors will still be in $S$. I got stuck here, but I have an idea where we prove that $\lambda_1x_1+\lambda_2x_2$ is in $S$ and then use induction to generalize to a list of $m$ scalars and $m$ vectors.

\textbf{1.24 Solution.} (a) If we take vector addition to be coordinate-wise addition, then it is a homomorphism: $F(x+x',y+y',z+z')=(z+z',y+y')=(z,y)+(z',y')=F(x,y,z)+F(x',y',z')$.

(b) Not a homomorphism: $F(x+x',y+y',z+z')=(x+x'+1,y+y',z+z'+1)\neq(x+x'+2,y+y',z+z'+2)=(x+1,y,z+1)+(x'+1,y',z'+1)=F(x,y,z)+F(x',y',z')$.

(c) Not a homomorphism:
$F(x+x',y+y')=(xy+xy'+x'y+x'y',0)\neq(xy+x'y',0)=(xy,0)+(x'y',0)=F(x,y)+F(x',y')$.

(d) Is a homomorphism:
$F(x+x',y+y')=(2x+2x'+3y+3y',x+x'-y-y')=(2x+3y,x-y)+(2x'+3y',x'-y')=F(x,y)+F(x',y')$.

\textbf{1.25 Solution.} We showed earlier that if $T\in Hom(U,V)$, then $\lambda T\in Hom(U,V)$ since $Hom(U,V)$ is a subspace of $\mathcal{F}(U,V)$. Then we only need to show that there exists a linear mapping $(\lambda T)^{-1}$ that takes $V$ back to $U$. To do this, we make use of the fact that $T$ is an isomorphism, and therefore there $T^{-1}$ is a homomorphism that satisfies additivity and homogeneity.

If we define $(\lambda T)^{-1}$ as $\lambda(T^{-1})$, then it satisfies the homomorphic properties of additivity and homogeneity:

In particular, we have $(\lambda T)^{-1}(v+v')=\lambda(T^{-1}(v)+T^{-1}(v'))=(\lambda T)^{-1}(v)+(\lambda T)^{-1}(v')$. So additivity holds.

Additionally, for some scalar $a\in\mathbb{F}$, we have $(\lambda T)^{-1}(av)=\lambda(T^{-1}(av))=\lambda(aT^{-1}(v))=a(\lambda T^{-1}(v))$. So homogeneity holds as well.

Therefore $\lambda T$ must also be an isomorphism.

\textbf{1.26 Solution.} In $\mathbb{C}_\mathbb{R}$ scalar multiplication essentially "scales" vectors in $\mathbb{C}$ by a real $a\in\mathbb{R}$. On the other hand, in $\mathbb{C}_\mathbb{C}$, scalar multiplication is actually "vector multiplication", in the sense that resulting vector is the result of multiplying two complex numbers.

First we show that the map $M:\mathbb{C}_\mathbb{R}\to \mathbb{R}^2$ is bijective. Since $a+bi$ maps to $(a,b)$, we can easily define an inverse mapping $(a,b)\to a+bi$. Therefore there is a one-to-one correspondence between $\mathbb{C}_\mathbb{R}$ and $\mathbb{R}^2$. Now all we have to show is that $M$ is a homomorphism. 

Since $M((a+bi)+(a'+b'i))=(a+a',b+b')=(a,b)+(a',b')=M(a+bi)+M(a'+b'i)$, $M$ satisfies additivity. Furthermore, since $M(\lambda(a+bi))=M(\lambda a+\lambda bi)=(\lambda a, \lambda b) = \lambda (a, b)=\lambda M(a+bi)$ for $\lambda\in\mathbb{R}$, $M$ also satisfies homogeneity. Therefore $M$ is an isomorphism.

Now, we will show that $M':\mathbb{C}_\mathbb{C}\to\mathbb{R}^2$ satisfies additivity but not homogeneity.

Since $M'((a+bi)+(a'+b'i))=(a+a',b+b')=(a,b)+(a',b')=M'(a+bi)+M'(a'+b'i)$, $M'$ satisfies additivity. If $z=c+di\in\mathbb{F}$, then we have $M'(z(a+bi))=(ac-bd, ad+bc)$. However, $zM'(a+bi)$ is invalid because we cannot multiply a coordinate pair $(a,b)\in\mathbb{R}^2$ by a complex scalar $z\in\mathbb{C}$, so $M'$ doesn't satisfy homogeneity.

\textbf{1.27 Solution.} %Call the complex conjugation map $M$. Then $M((a+bi)+(a'+b'i))=M((a+a')+(b+b')i)=(a+a')-(b+b')i=(a-bi)+(a'-b'i)=M(a+bi)+M(a'+b'i)$, so $M$ satisfies additivity. \textbf{Note: I got sort of stuck here... } Now, homogeneity can only hold if we define $M(z(a+bi))=M(z)M(a+bi)$ for all scalars $z\in\mathbb{C}$. That is, let $z=c+di$ be a scalar, then $M((c+di)(a+bi))=M((ac-bd)+(ad+bc)i)=(ac-bd)-(ad+bc)i=(c-di)(a-bi)=M(z)M(a+bi)$ (Notice that it does not equal to $z(M(a+bi))$!).
For $z=a+bi\in\mathbb{C}$, $z\cdot\overline{z}=a^2+b^2$.

\textbf{1.28 Solution.} First, we show that for two complex numbers $z,w$,  $\overline{z+w}=\overline{z}+\overline{w}$. This is fairly easy to show, as if we let $z=a+bi$ and $w=c+di$, then we have $\overline{z+w}=\overline{a+bi+c+di}= a+c-(b+d)i= (a-bi)+(c-di)=\overline{a+bi}+\overline{c+di}$.

Next, we show that $\overline{cz}=c\overline{z}$ for some scalar $c$. This is also fairly easy to show, as we have $\overline{c(z)}=\overline{ca+cbi}=ca-cbi=c(a-bi)=c\overline{z}$.

Finally, we show that $\overline{zw}=\overline{z}\cdot\overline{w}$. This is just algebra, and we get $\overline{zw}=\overline{(ac-bd)+(ad+bc)i}=(ac-bd)-(ad+bc)i=(a-bi)(c-di)=\overline{a+bi}\cdot\overline{c+di}$.

Now, let $z$ be a complex root to a polynomial $c_nx^n+c_{n-1}x^{n-1}+\ldots+c_0$ for scalars $c_i$. Then $c_nz^n+c_{n-1}z^{n-1}+\ldots+c_0=0$. Taking the conjugate of the entire expression and using what we proved above, we get:
\begin{align*}
0 &= \overline{c_nz^n+c_{n-1}z^{n-1}+\ldots+c_0} \\ &= \overline{c_nz^n}+\overline{c_{n-1}z^{n-1}}+\ldots+\overline{c_0} \\ &= c_n\overline{z^n}+c_{n-1}\overline{z^{n-1}}+\ldots+c_0 \\ &= c_n\overline{z}^n+c_{n-1}\overline{z}^{n-1}+\ldots+c_0
\end{align*}

Thus the conjugate $\overline{z}$ must also be a root of the polynomial.

\textbf{1.29 Solution.} Let $v,v'$ be two vectors in $ker(T)$ and $a$ a scalar in $\mathbb{F}$. Then by additivity of $T$, we have that $T(v+v')=T(v)+T(v')=\textbf{0}+\textbf{0}=\textbf{0}$, which means the vector $v+v'\in ker(T)$ and so $ker(T)$ is closed under addition. Furthermore by the homogeneity of $T$, we have that $T(av)=aT(v)=a\textbf{0}=\textbf{0}$, which means the vector $av\in ker(T)$ and so $ker(T)$ is closed under scalar multiplication as well. Thus $ker(T)$ is a subspace of $V$. 

If $U$ is a subspace of $V$, then if we have vectors $u,u'\in U$ such that $T(u),T(u')\in ran(T)$, then by additivity of $T$ we know that $T(u+u')\in ran(T)$ so therefore $T(u)+T(u')\in ran(T)$. Furthermore, if we have a scalar $a\in\mathbb{F}$, then since $U$ is closed under scalar multiplication and by the homogeneity of $T$, we have that $T(au)\in ran(T)$ so therefore $aT(u)\in ran(T)$. Thus $ran(T)$ is closed under addition and scalar multiplication and is a subspace of $W$.

\textbf{1.30 Solution.} Omitted.

\textbf{1.31 Solution.} First we show that for any linear map $T:V\to W$, $T(\textbf{0})=\textbf{0}$. Due to homogeneity of $T$, we have $T(\textbf{0})=T(0\cdot v)=0\cdot T(v)=\textbf{0}$, for $v\in V$.

Now, for a vector $u\in ker(T)$, we have 
\begin{align*}
w &= T(v_0) \\
    &= T(v_0 + \textbf{0}) \\
    &= T(v_0) + T(0) \\
    &= T(v_0) + \textbf{0} \\
    &= T(v_0) + T(u) \\
    &= T(v_0+u),
\end{align*}
so then we have $T^{-1}(w)=v_0+u$. This is equivalent to the affine subset $v_0+ker(T)=\{v_0+u\mid u\in ker(T)\}$. Therefore any solution to the equation $T(v)=w$ is of the form $v_0+u$ for $u\in ker(T)$.

\textbf{1.32 Solution.} The matrix
$\left(\begin{tabular}{cccc}
    1 & 0 & 0 & 0 \\
    0 & 0 & 0 & 1
\end{tabular}\right)$
takes $(x_1,x_2,x_3,x_4)\mapsto(x_1,x_4)$

\textbf{1.33 Solution.} The matrix associated with this homomorphism is $aI$, or:
\begin{center}$\left(\begin{tabular}{cccc}
    a & 0 & $\ldots$ & 0 \\
    0 & a & $\ldots$ & 0 \\
    $\vdots$ & $\vdots$ &  & $\vdots$ \\
    0 & 0 & $\ldots$ & a
\end{tabular}\right)$\end{center}

\textbf{1.34 Solution.} (a) Subtracting $I$ from both sides, we get $A^2+2A=-I$. Since matrix multiplication is distributive, we get $A(A+2I)=-I$. Now, multiplying both sides by $-I$, we get $A(-A-2I)=I$, therefore the inverse of $A$ exists and is $-A-2I$.

(b) $A^n$ for $n\in\mathbb{N}$ is the 2 by 2 matrix $\left(\begin{tabular}{cc}
    1 & $n\cdot a$ \\
    0 & 1
    \end{tabular}\right)$. The inverse of $A$ is the 2 by 2 matrix $\left(\begin{tabular}{cc}
    1 & -a \\
    0 & 1
    \end{tabular}\right)$.
    
\textbf{1.35 Solution.} If a matrix $A$ is invertible, then there exists a matrix $B$ such that $AB=BA=I$. Then for the linear map that it determines $T:V\to W$, if we have a vector $v\in V$, $Av=w$ for some $w\in W$, but since $Bw=BAv=Iv=v$, the inverse $T^{-1}$ is just the matrix $B$, thus the linear mapping $T$ is invertible.

Now, assume $T:V\to W$ is an invertible linear mapping. Then there exists an inverse linear mapping $T^{-1}:W\to V$. Say $T$ can be represented by the matrix $A$. Then since $T$ is invertible, let $B$ be the matrix that represents $T^{-1}$, then we have that $v=T^{-1}(T(v))=BAv$. Thus $A$ is invertible since there exists a matrix $B$ such that $BA=I$.

\textbf{1.36 Solution.} The inverse of $A$ is the n by n matrix:
\begin{center}
$\left(
\begin{tabular}{cccc}
    $\frac{1}{a_1}$ & 0 & $\ldots$ & 0 \\
    0 & $\frac{1}{a_2}$ & $\ldots$ & 0 \\
    $\vdots$ & $\vdots$ &  & $\vdots$ \\
    0 & 0 & $\ldots$ & $\frac{1}{a_n}$
\end{tabular}
\right)$
\end{center}
\newpage
\textbf{1.37 Solution.} The kernel of $D^n$ is the set of functions whose $n^{\text{th}}$ derivative is the 0 function, $\{f\mid f\in C^{\infty}(\mathbb{R},\mathbb{R}), f^n=0\}$. This consists of all the polynomials of degree less than $n$.

If $T=D^1-Id$, then for a function $f\in C^{\infty}(\mathbb{R},\mathbb{R})$, $T(f)=D^1(f)-Id(f)=f'-f$. The kernel of $T$ is then the set of all function $f$ such that $f'-f = 0$, or $f'=f$. This is just the set $\{f(x)=a\cdot e^x\mid x\in\mathbb{R},a\in\mathbb{F}\}$.

\textbf{1.38 Solution.} Suppose we have two sequences $z=(z_1,z_2,z_3,\ldots)$ and $w=(w_1,w_2,w_3,\ldots)$ such that $z,w\in V$. Then using coordinate wise addition we have $T(z+w)=T(z_1+w_1,z_2+w_2,z_3+w_3,\ldots)=(z_2+w_2,z_3+w_3,\ldots)=(z_2,z_3,\ldots)+(w_2,w_3,\ldots)=T(z_1,z_2,z_3,\ldots)+T(w_1,w_2,w_3,\ldots)=T(z)+T(w)$, satisfying additivity. Furthermore, if we have a scalar $a\in\mathbb{F}$, then by coordinate wise scalar multiplication we have $T(az)=T(az_1,az_2,az_3,\ldots)=(az_2,az_3,\ldots)=a(z_2,z_3,\ldots)=aT(z)$, satisfying homogeneity.

\textbf{1.39 Solution.} For any pair $(x,y)\in\mathbb{R}^2$, define $T:(x,y)\mapsto(0,y)$ and $S:(x,y)\mapsto(y,0)$. Then $(T\circ S)(x,y)=T(y,0)=(0,0)$, but $(S\circ T)(x,y)=S(0,y)=(y,0)\neq(0,0)$.

\textbf{1.40 Solution.} First, we know that every field is a vector space over itself. Then it follows that every field $\mathbb{F}$ is an algebra when we take the vector space as $\mathbb{F}$ over the field $\mathbb{F}$ with the binary operations of addition and multiplication carried over from addition and multiplication in $\mathbb{F}$. We know that the algebra must be associative because of the associativity of field multiplication, so it is an associative algebra. Furthermore, we know that the multiplicative identity exists and is the same element 1 from $\mathbb{F}$. Thus every field is a unital associative algebra.

\textbf{1.41 Solution.} Since $U$ is an isomorphism, there exists an inverse $U^{-1}:W\to V$ such that $UU^{-1}=U^{-1}U=Id$. If for every $T\in End(V)$, $F$ takes $T$ to a unique $UTU^{-1}\in End(W)$, then we can reverse map every $UTU^{-1}\in End(W)$ to a unique $T'\in End(V)$ with the mapping $UTU^{-1}\mapsto U^{-1}(UTU^{-1})U=T$. But then $T'=T$ and $F$ is a bijection. Now we only need to prove that $F$ is a homomorphism that takes an endomorphism in $V$ to an endomorphism in $W$.

Suppose $T:V\to V$ so that $T\in End(V)$. Then if we have vectors $v,v'\in V$ and $w,w'\in W$ such that $T(v)=v'$, $U(v)=w$, and $U(v')=w'$, we have that $UTU^{-1}(w)=UT(v)=U(v')=w'$. Therefore $UTU^{-1}$ sends each vector $w\in W$ to a unique vector $w'\in W$, so $UTU^{-1}\in End(W)$.

Now, for $T,T'\in End(V)$, we have $F(T+T')=U(T+T')U^{-1}=U(TU^{-1}+T'U^{-1})=UTU^{-1}+UT'U^{-1}=F(T)+F(T')$, satisfying additivity. Furthermore, for $a\in\mathbb{F}$, we have $F(aT)=U(aT)U^{-1}=a(UTU^{-1})=aF(T)$, satisfying homogeneity.

Thus $T\mapsto UTU^{-1}$ determines an isomorphism $F:End(V)\to End(W)$.

\textbf{1.42 Solution.} Since $A$ and $B$ are isomorphic, suppose we have $f:A\to B$ such that $f(a\cdot a')=f(a)\cdot f(a')$. Then for all $a_1,a_2,a_3\in A$ and $b_1,b_2,b_3\in B$, we have that $b_1\cdot(b_2\cdot b_3) = f(a_1)\cdot(f(a_2)\cdot f(a_3))= f(a_1\cdot(a_2\cdot a_3))=f((a_1\cdot a_2)\cdot a_3)=(f(a_1)\cdot f(a_2))\cdot f(a_3)= (b_1\cdot b_2)\cdot b_3$. Therefore due to the bijectivity of $f$, the algebra $B$ is associative.

Similarly, if replacing the word associative for unital, suppose we have the multiplicative identity $1_A\in A$ and define $1_B=f(1_A)$. Then for all $a\in A$ and $b\in B$ such that $f(a)=b$, we have $b\cdot 1_B = f(a)\cdot f(1_A) = f(a\cdot 1_A)=f(a)=b$ and also $1_B\cdot b = f(1_A)\cdot f(a) = f(1_A\cdot a)=f(a)=b$. Therefore we have $1_B\cdot b = b = b \cdot 1_B$, and thus $B$ is unital.

\textbf{1.43 Solution.} 

$(AB)C =
\left[\begin{tabular}{cc}
    2 & 2 \\
    -1 & 13
\end{tabular}\right]
\left[\begin{tabular}{c}
    3 \\
    1
\end{tabular}\right] =
\left[\begin{tabular}{c}
    8 \\
    10
\end{tabular}\right]$

$A(BC) = 
\left[\begin{tabular}{ccc}
    1 & 2 & -1 \\
    1 & 3 & 2
\end{tabular}\right]
\left[\begin{tabular}{c}
    4 \\
    2 \\
    0
\end{tabular}\right] =
\left[\begin{tabular}{c}
    8 \\
    10
\end{tabular}\right]$

\textbf{1.44 Solution.}

$A^2 = 
\left[\begin{tabular}{ccc}
    0 & 0 & 1 \\
    0 & 0 & 0 \\
    0 & 0 & 0
\end{tabular}\right]$

$A^3 = AA^2 = 
\left[\begin{tabular}{ccc}
    0 & 1 & 1 \\
    0 & 0 & 1 \\
    0 & 0 & 0
\end{tabular}\right]
\left[\begin{tabular}{ccc}
    0 & 0 & 1 \\
    0 & 0 & 0 \\
    0 & 0 & 0
\end{tabular}\right] =
\left[\begin{tabular}{ccc}
    0 & 0 & 0 \\
    0 & 0 & 0 \\
    0 & 0 & 0
\end{tabular}\right]$

\textbf{1.45 Solution.} $AX_i$ is the $m\times 1$ matrix taken from the $i^{\text{th}}$ column of $A$. For example, if $A = 
\left[\begin{tabular}{ccc}
    1 & 2 & 3 \\
    4 & 5 & 6 \\
    7 & 8 & 9
\end{tabular}\right]$, then $AX_1 = 
\left[\begin{tabular}{c}
    1 \\
    4 \\
    7
\end{tabular}\right]$,
$AX_2 = 
\left[\begin{tabular}{c}
    2 \\
    5 \\
    8
\end{tabular}\right]$, and
$AX_3 =
\left[\begin{tabular}{c}
    3 \\
    6 \\
    9
\end{tabular}\right]$.

\textbf{1.46 Solution.} $A^n = 
\left[\begin{tabular}{cc}
    cos(n$\theta$) & -sin(n$\theta$) \\
    sin(n$\theta$) & cos(n$\theta$)
\end{tabular}\right]$ for all real $\theta$ and $n\in\mathbb{Z}$. This can be thought of more simply as the rotational matrix of $\theta$, and so each application of $A$ just rotates by an additional $\theta$.

\textbf{1.47 Solution.} Since $A$ and $B$ are nilpotent matrices, suppose $A^x=B^y=0$ for integers $x,y>0$. 

Then by commutativity of $A$ and $B$, we have $(AB)^{xy}=(AB)(AB)\ldots(AB)=(A)^{xy}(B)^{yx}=(A^x)^y(B^y)^x=0^y0^x=0$. Thus $AB$ is nilpotent.

To prove $A+B$ is nilpotent, we raise $A+B$ to the $(x+y)^{\text{th}}$ power to get $(A+B)^{x+y} = A^{x+y}+\binom{x+y}{1}(A^{x+y-1})(B)+\ldots+ \binom{x+y}{x+y-1}(A)(B^{x+y-1})+B^{x+y}$, the binomial expansion (note that we can do this because matrix multiplication is distributive and associative). Now, we can see that each term of the binomial expansion can be written as $\binom{x+y}{i}(A^{x+y-i})(B^{i})$ for $0\leq i\leq x+y$. Then every term has either a power of $A$ at least $x$ or a power of $B$ at least $y$. Thus each term evaluates to 0, and so the sum evaluates to 0. Therefore $A+B$ is nilpotent as well.

\textbf{1.48 Solution.} Since $X$ is a set, each element must be distinct. It follows that each of the $x_i$ are distinct such that exactly one of the $\delta_{x_i}$ can evaluate to 1 and the rest to 0. Since $\{x_1,\ldots,x_k\}$ are the base points of $f$, we know that $f$ can be written as $f(x)\delta_{x_1} + f(x)\delta_{x_2} + \ldots + f(x)\delta_{x_k}$, so that $f$ evaluates to $f(x_i)$ when $x=x_i$ (since for all other $x_j\neq x_i$, $\delta_{x_j}$ evaluates to 0). Then this is just the summation $\sum_{i=1}^ka_i\delta_{x_i}$ where $a_i=f(x_i)$.

\textbf{1.49 Solution.} \boxed{\text{True.}} Since the vector space is $\mathbb{R}\langle\mathbb{R}\rangle$, we define vector addition (function addition) for $f,g\in\mathbb{R}\langle\mathbb{R}\rangle$ as $(f+g)(x)=f(x)+g(x)$. Similarly we define vector multiplication (function multiplication) as $(f\cdot g)(x)=f(x)g(x)$. Then we can deduce that $\mathbb{R}\langle\mathbb{R}\rangle$ is closed under both vector addition and multiplication since they simplify to field ($\mathbb{R}$) addition and multiplication. Furthermore, it is easy to check that vector multiplication is bilinear as vector multiplication and addition just reduce to multiplication in the field $\mathbb{R}$. Thus $\mathbb{R}\langle\mathbb{R}\rangle$ is an algebra over $\mathbb{R}$ under function addition and multiplication.

\textbf{1.50 Solution.} Suppose we take the vector space $\mathbb{R}$ over the field $\mathbb{R}$. Then if we have the subset of integers $\mathbb{Z}\subseteq\mathbb{R}$, we know that $\mathbb{Z}$ is closed under addition (adding any two integers returns another integer) but not scalar multiplication (multiplication by a fractional scalar does not produce an integer vector).

Now suppose we take the vector space $\mathbb{C}$ over the field $\mathbb{R}$. Then if we have the union of the real and imaginary axes, which is a subset of $\mathbb{C}$, we can see that it is closed under scalar multiplication (every scalar multiple of 1 and i are in our subset), but not closed under addition (e.g. the complex number 1+i is not in our subset).

A set closed under addition and scalar multiplication that is not a vector space is the empty set $\{\}$.

\textbf{1.51 Solution.} First, it is easy to see that for any $a,b\in\mathbb{R}^+$, $ab>0$ so $ab\in\mathbb{R}^+$, and therefore the set is closed under "addition". Furthermore, for any $a\in\mathbb{R}^+$ and $\frac{p}{q}\in\mathbb{Q}$, we know that $a^{\frac{p}{q}}>0$, so the set is also closed under "scalar multiplication".

Now, it is easy to see that $(\mathbb{R}^+,\boxplus)$ is an abelian group, since we have:
\begin{itemize}
    \setlength{\parskip}{0pt}
    \item ab=ba, so $\boxplus$ is commutative.
    \item a(bc)=(ab)c, so $\boxplus$ is associative.
    \item the identity exists and is $1\in\mathbb{R}^+$.
    \item the inverse of every element $a\in\mathbb{R}^+$ exists and is equivalent to $\frac{1}{a}\in\mathbb{R}^+$.
\end{itemize}

Now we confirm the remaining four axioms of a vector space for $\frac{p}{q},\frac{p'}{q'}\in\mathbb{Q}$ and $a,b\in\mathbb{R}^+$:
\begin{itemize}
    \setlength{\parskip}{0pt}
    \item $(a^{\frac{p}{q}})^{\frac{p'}{q'}} = a^{(\frac{p}{q}\cdot\frac{p'}{q'})}$, satisfying associativity of scalar multiplication.
    \item The identity of scalar multiplication exists and is the element $1\in\mathbb{Q}$, such that for every $a\in\mathbb{R}^+$, $a^1=a$.
    \item $a^{(\frac{p}{q}+\frac{p'}{q'})} = a^{\frac{p}{q}}a^{\frac{p'}{q'}}$, satisfying distributivity over scalar addition.
    \item $(ab)^{\frac{p}{q}}=a^{\frac{p}{q}}b^{\frac{p}{q}}$, satisfying distributivity over vector addition.
\end{itemize}

Thus we have shown that the set is indeed a vector space. The zero vector in $\mathbb{R}^+$ is the number $1$ since any positive real "added" to $1$ is the same as multiplying (classic multiplication) by $1$.

\textbf{1.52 Solution.} First, it is easy to see that the power set of $X$ is closed under $\Delta$, as any set taken from the exclusive or of two subsets of $X$ is still a subset of $X$. Furthermore, $\textbf{P}(X)$ is also closed under scalar multiplication; if we define $0S=\{\}$ and $1S=S$.

Now, we show that $(\textbf{P}(X),\Delta)$ is an abelian group:
\begin{itemize}
    \setlength{\parskip}{0pt}
    \item Exclusive Or is clearly commutative, since it is the set of all elements in either $A\subseteq X$ or $B\subseteq X$ but not both.
    \item Exclusive Or is also associative, since both $A\Delta(B\Delta C)$ and $(A\Delta B)\Delta C$ both evaluate to the set of all elements in exactly one of $A$, $B$, and $C$.
    \item The identity element exists and is the empty set $\{\}$.
    \item For every $A\in\textbf{P}(X)$, the inverse element exists and is $A$ itself.
\end{itemize}

Next, we confirm the remaining four axioms of a vector space for all $A,B\in\textbf{P}(X)$ and $x,y\in\mathbb{Z}_2$:
\begin{itemize}
    \setlength{\parskip}{0pt}
    \item It is easy to check that for all combinations of $x,y\in\mathbb{Z}_2$ ($(0,0),(0,1),(1,0),(1,1)$), scalar multiplication is associative. That is, $x(yA) = (xy)A$.
    \item The identity of scalar multiplication exists and is defined to be $1\in\mathbb{Z}_2$, such that $1A=A=A1$.
    \item $(0+0)A = \{\} = (0A)\Delta(0A)$, $(1+0)A = A = (1A)\Delta(0A)$, $(1+1)A = \{\} = (1A)\Delta(1A)$, and $(0+1)A = A = (0A)\Delta(1A)$, satisfying distributivity over scalar addition.
    \item $0(A+B) = \{\} = (0A)\Delta(0B)$, and $1(A+B) = A\Delta B = (1A)\Delta(1B)$, satisfying distributivity over vector addition.
\end{itemize}
Therefore $\textbf{P}(X)$ over the field $\mathbb{Z}_2$ under $\Delta$ is indeed a vector space.

\textbf{1.53 Solution.} We will show that $\textbf{P}(\{p\})$ is the underlying set of a vector space over $\mathbb{Z}_2$ when embellished with Exclusive Or as vector addition.

Since $\{p\}\subseteq\textbf{P}(X)$, from our results in exercise 1.52, all we have to show is that the set is closed under Exclusive Or and scalar multiplication.

We have that $\{p\}\Delta\{\}=\{p\}$, $\{p\}\Delta\{p\}=\{\}$, and $\{\}\Delta\{\}=\{\}$, so the set is closed under vector addition. Finally, we have that $1\{p\}=\{p\}$, $0\{p\}=\{\}$, $1\{\}=\{\}$, and $0\{\}=\{\}$, therefore the set is also closed under scalar multiplication.

Thus, $\textbf{P}(\{p\})$ is the underlying set of a vector space, more specifically a subspace of $\textbf{P}(X)$.


